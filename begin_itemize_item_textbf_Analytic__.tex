\begin{itemize}
\item \textbf{Analytic}: Analyticity is the property for a function to be \verb|differentiable| at \verb|all points| in a certain domain (i.e., not only at one point).
\begin{quote}
A function $f(z)$ is said to be analytic \verb|at a point $z$| if $z$ is an \verb|interior point| of some region where $f(z)$ is analytic.
\end{quote}
\end{itemize}
\verb|(b) Evaluation of Analytic functions|
\begin{itemize}
\item \verb|Cauchy-Riemann equation| (necessary condition): If $f(z) = u(x,y) + iv(x,y)$ is analytic in some domain, then 
$$
u_x = v_y\ \mathrm{and}\ u_y = -v_x
$$ in that domain. 
\item \verb|Necessary and sufficient conditions|: Provided that all these partial derivatives are \verb|continuous in the domain| and satisfy the C-R equation.
\item \verb|Harmonic functions|: If $f(z)$ is analytic in some domain, then in that domain
$$
\nabla^2u = 0\ \mathrm{and}\ \nabla^2v = 0.
$$ where
$$
\nabla^2 g(x,y) = \frac{\partial^2 g}{\partial x^2} + \frac{\partial^2 g}{\partial y^2}.
$$ Contrariwise, if either $\mathrm{Re}\{f(z)\}$ or $\mathrm{Im}\{f(z)\}$ does not satisfy Laplace's equation, then $f(z)$ is not analytic.
\end{itemize}
\verb|(c) Properties|
\begin{itemize}
\item \textbf{Connection with continuous}: If $f(z)$ is analytic at a point $z$, then the derivative $f'(z)$ is \verb|continuous| at $z$; and further it has continuous derivatives of \verb|all order| at the point $z$.
\item \textbf{Singular points}: Points at which a function $f(z)$ is \textbf{not analytic} are called singular points. There are three different types of singular points:
\begin{enumerate}
\item \textit{Removable singularities}: $f$ is defined in a neighborhood of the point $z$, but not at $z$, but $f$ can be defined at $z$ so that $f$ is a continuous function which includes $z$. Example: $f(z) = z, z \in \mathcal{C} \setminus \{0\}$. That means $z=0$ is a removable singularity; $f(z) = \mathrm{sin} z/z$ since the limit is 1 as $z$ approaches 0.
\item \textit{Pole}: $f$ blows up at $z$ ($f$ goes to infinity as approaching $z$). Example: $z=0$ for $f(z) = 1/z$.
\item \textit{Essential singularity}: The limit as $f$ approaches $z$ takes on different values as approaching $z$ from different directions. 
\end{enumerate}
\end{itemize}
